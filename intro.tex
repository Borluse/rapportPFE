\section{Introduction}

\subsection{Présentation du sujet} % (fold)
\label{sub:présentation_du_sujet}

% subsection présentation_du_sujet (end)

Suivant le developpement des mobiles, nous pouvons faire plus de chose qu'avant. Les mobiles aujourd'huis ne sont pas seulement un appareil pour téléphoner, mais aussi ont la capable de réaliser des functions qu'on ne peut pas imaginer avant. Les mobiles les plus modernes aujourd'huis sont les mobiles qui s'appele Smart-Phone. 

Le smart-phone est un genre de téléphone qui a un système d'exploitation dedans. Le smart-phone permet d'exécuter les applications écrit par les developpeurs. Le smart-phone est en fait un ordinateur spéciale en size. Son architecture ressemble à l'architecture d'ordinateur. Il a compris un CPU, des mémoire, une carte de mère, souvent des lecteurs, etc.. Et aujourd'hui, le téléphone le plus moderne a aussi une carte 3D, un module de WIFI, un module 3G, un module de GPS, un appareil de photo etc. Cela nous permet de créer beaucoup des applications intéressant et puissant. 

Mon Projet de fin d'étude est réalisé sur le smart-phone. 

Le sujet de mon projet de fin d'étude est \guillemotleft Reconnaissance de bâtiment \guillemotright, proposé par Melle Sabine Barrat. Ce projet signifie à créer un système sur le smart-phone qui peut obtenir automatiquement les informations sur un bâtiment prise en photo par le smart-phone. Les informations peuvent être l'adresse du bâtiment, le nom, le fonctionnalité, etc. Pour ce projet de fin d'étude, deux objectifs sont visés : 
\begin{enumerate}
	\item L’utilisateur se trouve devant un bâtiment et le prend en photo en vue d’obtenir des informations dessus. 
	\item L’utilisateur peut également prendre en photo une affiche ou une publicité présentant un bâtiment qui l’intéresse.
\end{enumerate}

Ce projet de fin d'étude est divisé en trois approche :

\begin{enumerate}
	\item Dans un premier temps, il faut essayer de retirer les points d'intérêt d'une photo. Nous avons 2 algorithmes à choisi, soit l'algorithme SIFT, soit l'algorithme SURF. Les calcules des points d'intérêts vont être réalisés sur le smart-phone et le serveur. 
	\item Après avoir des points d'intérêts, nous allons faire la comparaison des points d'intérêts entre la photo de requête et les photos dans la base de donnée. Cette partie va trouver la plus proche photo.
	\item Une fois on a trouver la plus proche photo, nous pouvons retirer ses information du serveur, et les afficher sur le smart-phone. Les informations sont enregistré sous forme d'un fichier XML. Nous pouvons rechercher ces informations grâce au moteur de recherche GOOGLE.
\end{enumerate}

L'architecture de ce projet de fin d'étude est Server-Client. La base de donnée est installé sur le serveur, qui comprend une large quantité des images. 

\subsection{Smart-Phone choisi} % (fold)
\label{sub:smart_phone_choisi}

Pour ce projet de fin d'étude, il faut au début décider le Smart-Phone pour développer. Melle Sabine Barrat m'avait proposé deux choix :
\begin{description}
	\item[Android] Le Smart-Phone Android est les téléphones qui utilisent Google Android comme leur système d'exploitation. L'application sur Android est du  Java.
	\item[iPhone] Le Smart-Phone iPhone est développé par Apple. iPhone utilise le système d'exploitation iOS. iOS est un variation du système d'exploitation MacOS X. L'application sur iPhone est du langage Objective-C qui est un extension du langage C. 
\end{description}

Nous avons décidons de choisir iPhone comme le plateforme pour ce projet de fin d'étude. 
% subsection smart_phone_choisi (end)

\section{Outils Utilisé} % (fold)
\label{sec:bilan_personnel}

\subsection{XCode} % (fold)
\label{ssub:xcode}

XCode est une environnement de développement sur MacOS. Cet IDE (Environnement d Développement Intégré) est fournit par Apple. XCode est très bien intégré avec le framework iOS. Il fournit aussi plusieurs outils très convient. Donc il est plus facile d'utiliser XCode pour développer des applications sur iPhone. XCode fournit de nombreuses fonctionnalités : 
\begin{enumerate}
	\item Support multi langage : XCode peut être non seulement utilisé pour développer l'application du Objective-C sur iPhone, mais aussi utilisé pour développer l'application du C et C++. En plus, il peut supporter Java, PHP etc. 
	\item Intégré dans le système MacOS : XCode est crée par Apple. Grace à ça, XCode peut marcher plus vite, fournir le meilleur stabilité, fournir des outils spéciale pour iPhone etc..
	\item Auto-complétion : XCode fournit la fonctionnalité d'auto-complétion. Bien que beaucoup d'IDE fournit cette fonctionnalité, mais pour le développement de iPhone, XCode est le meilleur choix. 
	\item Gestion des projet : XCode permet de gérer les projets iOS très facilement. Il fournit des façons simple pour ajouter des frameworks, modifier des configurations etc.. 
\end{enumerate}
En résumé, pour développer l'application sur iPhone, XCode est le meilleur solution.
% subsubsection xcode (end)

\subsection{Git & github} % (fold)
\label{sub:github}

Git est un logiciel de gestion de versions décentralisée, crée par Linus Torvalds. Git ne repose pas sur un serveur centralisé. C'est un outil bas niveau, qui se veut simple et très performant. Git indexe les fichiers d'après leur somme de contrôle calculée avec la fonction SHA-1. Quand un fichier n'est pas modifié, la somme de contrôle. 

L'avantage de git est que il est très legère, mais puissant et rapide. Il a utilisé un méchanisme différent que les autres version contrôle.

Pour mieux utiliser Git, nous pouvons utiliser un serveur sur Internet pour enregistrer notre sources. Github (www.github.com) est un service web d'hébergement et de gestion de développement de logiciels utilisant Git. Aussi, git est un logiciel gratuit.


% subsection github (end)

\subsection{Gitti} % (fold)
\label{sub:gitti}

Gitti est un logiciel gratuit pour mieux utiliser git. Il support github aussi. Ce logiciel permet d'utiliser git facilement, sans mémoriser les commandes de git. Il peut représenter les branches en graphique aussi. Gitti suit le standard d'interface défini par Apple. Il est très facile à maître. 


% subsection gitti (end)

\subsection{MAMP} % (fold)
\label{sub:mamp}

Ce projet de fin d'étude comprend un coté de serveur. Pour simplifier ce projet de fin d'étude, j'ai décidé de utiliser mon MacOS comme la serveur. J'ai choisi le logiciel MAMP pour ce serveur.

MAMP est un acronyme informatique signifiant : 

\begin{description}
	\item[Macintosh] Le système d'exploitation, MacOS 
	\item[Apache] Le serveur web connu du monde.
	\item[MySQL] Le serveur de base de donnée.
	\item[PHP] PHP est un langages de script pour créer des applications sur la coté serveur.
\end{description}

Comme LAMP, MAMP est un des logiciels connu sur MacOS. Il peut installer les 4 composantes en même temps et bien configuré. C'est une solution convenable. 

% subsection gitti (end)

\subsection{CMake} % (fold)
\label{sub:mamp}

CMake est un 'Moteur de production' multiplate-forme. Il est comparable au programme Make dans le sens ou le processus de construction logicielle est entièrement contrôlé par un fichier de configuration, appelés CMakeLists.txt.Mais Cmake ne produit pas directement le code finale. CMake va se charger de configurer et générer le Makefile sous Unix. 

Le nom 'Cmake' signifie à 'Cross platform make'. Même si il y a un 'make' dans son nom, CMake lui-même est un application de haut niveau. 

Donc ce projet de fin d'étude, j'ai utilisé plusieur framework, par exemple OpenCV, libSurf etc. Les framework sont écrit originale sous système Linux. Pour pouvoir compilé sous MacOS et modifié pour le but d'intégrer dans le plateforme iOS. Il faut reconfigurer les flags, les options etc. Nous utilisons le premier fois CMake pour générer un projet XCode.  Et ensuite, nous utilisons XCode pour modifier et compiler ce projet généré. Et comme ça, nous obtenons un framework spécial pour le framework iOS.

% subsection CMakes (end)




% section bilan_personnel (end)



\section{Problème rencontré} % (fold)
\label{sec:bilan_personnel}


\subsection {OpenCV sur IOS}
\label{sec:openv_sur_ios}

Le gros problème est de comment réaliser l'algorithme SIFT ou SURF sur iOS. Les 2 algorithmes sont tout à fait difficile et complexe à réaliser. Et en plus, nous n'avons pas beaucoup de temps à réaliser. Donc la premier chose à faire est d'essayer de trouver une bibliothèque qui contient les 2 algorithmes. Après longtemps de recherche, j'ai trouvé une biblothèque qui s'appelle OpenCV. 

Après avoir configuré par CMake, nous obtenions une bibliothèque compilé sur iOS. Mais le problème est que il existe des problème quand j'ai essayé d'intéger OpenCV dans le code Objective-C. Cela est causé par l'architecture spécial de iPhone, et le framework spécial de iOS.



